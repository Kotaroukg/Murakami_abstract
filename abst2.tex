\documentclass[a4j,twocolumn,uplatex]{jsarticle}
%\documentclass[a4j,twocolumn,uplatex]{jsarticle}

\usepackage[dvipdfmx]{graphicx}

\setlength{\textheight}{275mm}
\headheight 5mm
\topmargin -30mm
\textwidth 185mm
\oddsidemargin -15mm
\evensidemargin -15mm
\pagestyle{empty}

\begin{document}
\title{Processingによるホイヘンスの原理の視覚化}
\author{情報科学科 西谷研究室3518 村上 大貴}
\date{}
\maketitle
\section{はじめに.}
物理現象は直感だと非常に理解しづらいものが存在する.
例えば,波の進み方の原理を説明するホイヘンスの原理は,実世界には存在しない素元波という概念を用いて波面を決定していく.しかし,実世界にないが故に直感的な理解を得るのは難しい.これを直感的に理解させ,学習を促進する手段として,物理現象の視覚化が適していると考える.
本研究では,素元波の発生と,波面の決定をプログラムにより視覚化することで,ホイヘンスの原理の理解を容易にし,この原理により説明できる,波の反射,屈折,回折を深く理解できるようなツールをProcessingにより作成することを目的とする.またこれをはじめとする物理現象の視覚化プログラムを容易に投稿,閲覧できるサイトを構築することも目指す.


\section{Processingについて.}
Processing(プロセッシング)とはオープンソースのプログラミング言語およびソフトウェア開発環境である.\cite{1}
Processing言語は以下に挙げるような特徴を有している.
1. 基本文法はJavaをベースに記法を簡単化したものであり,インタラクティブソフトウェアやビジュアルプレゼンテーションを容易に実現することに特化している.\cite{1}
2. 環境構築が非常に容易かつ無償で行える.
3. Windowa, Mac, Linuxのクロスプラットフォームで動作する.\cite{1}これらの理由から,本研究で使用するプログラミング言語に最も適していると考える.


\section{現状}






\section{今後の課題}
今回作成したプログラムでは波の反射,屈折,回折の視覚化を実装するまでには至っていなので,これを自作する.


\begin{thebibliography}{9}
\bibitem{1} 石川将人,北卓人,大須賀公一. Arduino/Processingを用いたシステム制御実験のラピッドプロトタイピング. 自動制御連合講演会講演論文集 53(0), 2010. p.247 

\end{thebibliography}
\end{document}

