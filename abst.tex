\documentclass[a4j,twocolumn,uplatex]{jsarticle}
%\documentclass[a4j,twocolumn,uplatex]{jsarticle}

\usepackage[dvipdfmx]{graphicx}

\setlength{\textheight}{275mm}
\headheight 5mm
\topmargin -30mm
\textwidth 185mm
\oddsidemargin -15mm
\evensidemargin -15mm
\pagestyle{empty}

\begin{document}
\title{GitHubを利用したRuby学習支援ソフトの開発}
\author{情報科学科 西谷研究室 2549 浦田 航貴}
\date{}
\maketitle
\section{はじめに}
「西谷研究室に在籍している学生は,
Rubyプログラミングを修得するために初心者向けのweb上の問題集を使って学習している.」(よりメタに)
さらに,進捗状況の管理や指導者からの添削をより容易におこなえるように改善するため,
バージョン管理ソフトGitHubを利用するシステム(rubynovice)を開発している.
そこでは,Rubyプログラミングで重要となるテスト駆動をおこなえる環境を提供している
これにより,学習者自身が出力チェックできるようにし,
Rubyプログラミングにおけるテスト実行に自然と慣れるような学習形態を目指している.
しかし,テスト実行において,初心者が混乱する状況となっている.

そこで,本研究ではRubyのテスティングフレームワークを変更することでこの混乱を回避することを目的としている.

\section{GitHubについて}
GitHubは,コンピュータープログラムの元となるソースコードをインターネット上で管理するためのサービスである.
複数人が携わるソフトウェア開発において,ソースコードの共有や,バージョン管理といった作業は必要不可欠となる.
またソースコードを始めとするプログラム開発に必要なファイルやそれらの変更履歴等を保存する「リポジトリ」と呼ばれる場所があり,ソースコード等のバージョンを管理する機能の他,プログラム開発等に対する開発者間でのレビューやコメント機能,プログラム開発の進捗を管理する機能等が備わっている\cite{1}.

\section{現状}
\subsection{テスティングフレームワーク}
RubynoviceのテスティングフレームワークとしてはRSpecを実装している.これは,プログラムの振舞いを記述するためのドメイン特化言語を提供するフレームワークであり,「プログラムの振舞い」とはプログラム全体あるいは様々なレベルでの部分 (モジュールやクラス,メソッド) に対して期待する振舞いのことである.またドメイン特化言語 (Domain Specific Language:DSL) は,特定の問題領域 (ドメイン) を記述するために設計された言語である\cite{2}.

Rspecによって期待されている値と出力している値が一致しているかを確認できる.テストコードを使えば「puts を使って毎回目視で確認」なんてするよりも,高速で確実に実行結果を検証することができる.また,テストコードを書いておけば他の人も「このメソッドを呼ぶと何が起こるか」を理解しやすくなる.

しかし簡単なRubyのコードを書く場合、わざわざRspec書くのは大げさではないかと考えた.簡単なテストコードassert\_equalが使えれば十分だと考えた.下記の表1はTest::UnitとRSpecを比較した表である.

\begin{table}[htb]
  \begin{center}
   \caption{Test::Unit vs RSpec}
   \begin{tabular}{|l|l|l|} \hline
        & Test::Unit  &  RSpec  \\
    \hline \hline
    シンタックス  & ピュアRuby  & DSL(ドキュメント風) \\
    修得ハードル  & Rubyの文法   &  DSLを覚える必要あり\\
    %ロジックの自由度  & Rubyの言語機能をフル活用  & DSLに制約されやすい \\
    検証メソッド  & assert \_ equal A B  &  expect(B) to eq A \\
    利用率,情報量&低い,少ない&高い多い\\
    Rubyとの関係 & Ruby標準バンドル& Ruby標準ではない\\
    Railsとの関係 & Rails標準&Rails標準ではない\\
    本体のコード&少ない,シンプル&多い,複雑\\
    \hline
  \end{tabular}
  \end{center}
\end{table}

修得ハードルにおいて,Test::UnitはRubyの文法なのに対してRSpecは,DSL(ドメイン固有言語)を覚えなければならないのでtestを作る者は学習コストが大きいと感じた.またTest::Unitは,Ruby標準バンドルであるという利点もある.
\section{結論と今後の課題}
現段階,西谷研究室に在籍している学生はテストする時,RSpecを使用している.RSpecは,独自のDSL(ドメイン固有言語)を使っているために学習コストが大きいと感じた.

また初心者として私自身がRSpecを使用したところ,少しの間違いでもエラーが複雑に表示されることや,すぐにエラーを理解するということが困難であった.より単純でわかりやすい結果を出力してくれるTest::Unit(minitest)を用いることを考えている\cite{3}.

\begin{thebibliography}{9}
\bibitem{1} 「GitHub」,横田一輝, https://kotobank.jp/word/GitHub-1725201
\bibitem{2} 「Rubyst Magazine」, かくたに もろはし vol.54 http://magazine.rubyist.net/?0021-Rspec
\bibitem{3} 「give IT a try」,伊藤淳一, http://blog.jnito.com/entry/2015/07/13/073458
\end{thebibliography}
\end{document}
